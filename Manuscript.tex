\documentclass[floatsintext,man]{apa6}

\usepackage{amssymb,amsmath}
\usepackage{ifxetex,ifluatex}
\usepackage{fixltx2e} % provides \textsubscript
\ifnum 0\ifxetex 1\fi\ifluatex 1\fi=0 % if pdftex
  \usepackage[T1]{fontenc}
  \usepackage[utf8]{inputenc}
\else % if luatex or xelatex
  \ifxetex
    \usepackage{mathspec}
    \usepackage{xltxtra,xunicode}
  \else
    \usepackage{fontspec}
  \fi
  \defaultfontfeatures{Mapping=tex-text,Scale=MatchLowercase}
  \newcommand{\euro}{€}
\fi
% use upquote if available, for straight quotes in verbatim environments
\IfFileExists{upquote.sty}{\usepackage{upquote}}{}
% use microtype if available
\IfFileExists{microtype.sty}{\usepackage{microtype}}{}

% Table formatting
\usepackage{longtable, booktabs}
\usepackage{lscape}
% \usepackage[counterclockwise]{rotating}   % Landscape page setup for large tables
\usepackage{multirow}		% Table styling
\usepackage{tabularx}		% Control Column width
\usepackage[flushleft]{threeparttable}	% Allows for three part tables with a specified notes section
\usepackage{threeparttablex}            % Lets threeparttable work with longtable

% Create new environments so endfloat can handle them
% \newenvironment{ltable}
%   {\begin{landscape}\begin{center}\begin{threeparttable}}
%   {\end{threeparttable}\end{center}\end{landscape}}

\newenvironment{lltable}
  {\begin{landscape}\begin{center}\begin{ThreePartTable}}
  {\end{ThreePartTable}\end{center}\end{landscape}}




% The following enables adjusting longtable caption width to table width
% Solution found at http://golatex.de/longtable-mit-caption-so-breit-wie-die-tabelle-t15767.html
\makeatletter
\newcommand\LastLTentrywidth{1em}
\newlength\longtablewidth
\setlength{\longtablewidth}{1in}
\newcommand\getlongtablewidth{%
 \begingroup
  \ifcsname LT@\roman{LT@tables}\endcsname
  \global\longtablewidth=0pt
  \renewcommand\LT@entry[2]{\global\advance\longtablewidth by ##2\relax\gdef\LastLTentrywidth{##2}}%
  \@nameuse{LT@\roman{LT@tables}}%
  \fi
\endgroup}


\ifxetex
  \usepackage[setpagesize=false, % page size defined by xetex
              unicode=false, % unicode breaks when used with xetex
              xetex]{hyperref}
\else
  \usepackage[unicode=true]{hyperref}
\fi
\hypersetup{breaklinks=true,
            pdfauthor={},
            pdftitle={Updating Aversive Racism for the US South in 2018},
            colorlinks=true,
            citecolor=blue,
            urlcolor=blue,
            linkcolor=black,
            pdfborder={0 0 0}}
\urlstyle{same}  % don't use monospace font for urls

\setlength{\parindent}{0pt}
%\setlength{\parskip}{0pt plus 0pt minus 0pt}

\setlength{\emergencystretch}{3em}  % prevent overfull lines


% Manuscript styling
\captionsetup{font=singlespacing,justification=justified}
\usepackage{csquotes}
\usepackage{upgreek}

 % Line numbering
  \usepackage{lineno}
  \linenumbers


\usepackage{tikz} % Variable definition to generate author note

% fix for \tightlist problem in pandoc 1.14
\providecommand{\tightlist}{%
  \setlength{\itemsep}{0pt}\setlength{\parskip}{0pt}}

% Essential manuscript parts
  \title{Updating Aversive Racism for the US South in 2018}

  \shorttitle{Aversive Racism Update}


  \author{Curtis E. Phills\textsuperscript{1}, Heather Truelove\textsuperscript{1}, ?\textsuperscript{1}, ?\textsuperscript{1}, \& ?\textsuperscript{1}}

  % \def\affdep{{"", "", "", "", ""}}%
  % \def\affcity{{"", "", "", "", ""}}%

  \affiliation{
    \vspace{0.5cm}
          \textsuperscript{1} University of North Florida  }

  \authornote{
    Funding for this project was supported by a Small Grant from the College
    of Arts and Sciences at the University of North Florida to the first
    author and an undergraduate research grant from the University of North
    Florida to the fourth author.
    
    Correspondence concerning this article should be addressed to Curtis E.
    Phills, 1 UNF Drive, Jacksonville, FL, 32224. E-mail:
    \href{mailto:curtis.phills@unf.edu}{\nolinkurl{curtis.phills@unf.edu}}
  }


  \abstract{Enter abstract here. Each new line herein must be indented, like this
line.}
  \keywords{Prejudice, Aversive Racism, Hiring Decisions \\

    \indent Word count: X
  }




  \raggedbottom

\usepackage{amsthm}
\newtheorem{theorem}{Theorem}[section]
\newtheorem{lemma}{Lemma}[section]
\theoremstyle{definition}
\newtheorem{definition}{Definition}[section]
\newtheorem{corollary}{Corollary}[section]
\newtheorem{proposition}{Proposition}[section]
\theoremstyle{definition}
\newtheorem{example}{Example}[section]
\theoremstyle{definition}
\newtheorem{exercise}{Exercise}[section]
\theoremstyle{remark}
\newtheorem*{remark}{Remark}
\newtheorem*{solution}{Solution}
\begin{document}

\maketitle

\setcounter{secnumdepth}{0}



The materials, data, analysis code, and pre-registration documents for
this project are shared on the Open Science Framework
(\url{https://osf.io/jxb73/}). The analyses in this paper used R
(Version 3.5.1; R Core Team, 2018) and the R-packages \emph{dplyr}
(Version 0.7.6; Wickham, François, Henry, \& Müller, 2018),
\emph{forcats} (Version 0.3.0; Wickham, 2018a), \emph{ggplot2} (Version
3.0.0; Wickham, 2016), \emph{papaja} (Version 0.1.0.9709; Aust \& Barth,
2018), \emph{purrr} (Version 0.2.5; Henry \& Wickham, 2018),
\emph{readr} (Version 1.1.1; Wickham, Hester, \& Francois, 2017),
\emph{stringr} (Version 1.3.1; Wickham, 2018b), \emph{tibble} (Version
1.4.2; Müller \& Wickham, 2018), \emph{tidyr} (Version 0.8.1; Wickham \&
Henry, 2018), and \emph{tidyverse} (Version 1.2.1; Wickham, 2017).

\section{Pilot Study}\label{pilot-study}

\subsection{Method}\label{method}

\subsubsection{Participants and Design}\label{participants-and-design}

\subsubsection{Procedure}\label{procedure}

\paragraph{Task 1}\label{task-1}

\paragraph{Task 2}\label{task-2}

\subsection{Results}\label{results}

\subsubsection{Hypothesis 1}\label{hypothesis-1}

\subsubsection{Hypothesis 2}\label{hypothesis-2}

\subsection{Discussion}\label{discussion}

\section{Main Study}\label{main-study}

\subsection{Method}\label{method-1}

\subsubsection{Participants and Design}\label{participants-and-design-1}

\subsubsection{Procedure}\label{procedure-1}

\paragraph{Task 1}\label{task-1-1}

\paragraph{Task 2}\label{task-2-1}

\subsection{Results}\label{results-1}

\subsubsection{Hypothesis 1}\label{hypothesis-1-1}

\subsubsection{Hypothesis 2}\label{hypothesis-2-1}

\subsection{Discussion}\label{discussion-1}

\section{General Discussion}\label{general-discussion}

\newpage

\section{References}\label{references}

\begingroup
\setlength{\parindent}{-0.5in} \setlength{\leftskip}{0.5in}

\hypertarget{refs}{}
\hypertarget{ref-R-papaja}{}
Aust, F., \& Barth, M. (2018). \emph{papaja: Create APA manuscripts with
R Markdown}. Retrieved from \url{https://github.com/crsh/papaja}

\hypertarget{ref-R-purrr}{}
Henry, L., \& Wickham, H. (2018). \emph{Purrr: Functional programming
tools}. Retrieved from \url{https://CRAN.R-project.org/package=purrr}

\hypertarget{ref-R-tibble}{}
Müller, K., \& Wickham, H. (2018). \emph{Tibble: Simple data frames}.
Retrieved from \url{https://CRAN.R-project.org/package=tibble}

\hypertarget{ref-R-base}{}
R Core Team. (2018). \emph{R: A language and environment for statistical
computing}. Vienna, Austria: R Foundation for Statistical Computing.
Retrieved from \url{https://www.R-project.org/}

\hypertarget{ref-R-ggplot2}{}
Wickham, H. (2016). \emph{Ggplot2: Elegant graphics for data analysis}.
Springer-Verlag New York. Retrieved from \url{http://ggplot2.org}

\hypertarget{ref-R-tidyverse}{}
Wickham, H. (2017). \emph{Tidyverse: Easily install and load the
'tidyverse'}. Retrieved from
\url{https://CRAN.R-project.org/package=tidyverse}

\hypertarget{ref-R-forcats}{}
Wickham, H. (2018a). \emph{Forcats: Tools for working with categorical
variables (factors)}. Retrieved from
\url{https://CRAN.R-project.org/package=forcats}

\hypertarget{ref-R-stringr}{}
Wickham, H. (2018b). \emph{Stringr: Simple, consistent wrappers for
common string operations}. Retrieved from
\url{https://CRAN.R-project.org/package=stringr}

\hypertarget{ref-R-tidyr}{}
Wickham, H., \& Henry, L. (2018). \emph{Tidyr: Easily tidy data with
'spread()' and 'gather()' functions}. Retrieved from
\url{https://CRAN.R-project.org/package=tidyr}

\hypertarget{ref-R-dplyr}{}
Wickham, H., François, R., Henry, L., \& Müller, K. (2018). \emph{Dplyr:
A grammar of data manipulation}. Retrieved from
\url{https://CRAN.R-project.org/package=dplyr}

\hypertarget{ref-R-readr}{}
Wickham, H., Hester, J., \& Francois, R. (2017). \emph{Readr: Read
rectangular text data}. Retrieved from
\url{https://CRAN.R-project.org/package=readr}

\endgroup






\end{document}
